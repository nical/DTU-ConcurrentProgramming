
\section{Step 4}

\subsection{Monitors} % Nico

The use of Java monitors brings some syntactic sugar to the implementation of
the mutual exclusion when accessing members of the alley/barrier/etc. The
algorithms can be rewritten by replacing:
\begin{verbatim}
[...]
atomicAccess.P();
[...] // access to private members
atomicAccess.V();
[...]
\end{verbatim}
with:
\begin{verbatim}
[...]
synchronized(this) {
[...]  // access to private members
}
[...]
\end{verbatim}

As shown in the monitor based implementations of the alleys and the barrier.
Monitors are convenient to ensure mutual exclusion but do not help with blocking
threads intentionally as the main semaphores of the barrier and the alley do.
Semaphores are better suited for this task, and are kept in the implementation
of the monitor based alley and barrier.

Using monitors follow the same basic rules that with mutual exclusion semaphores
(\texttt{atomicAccess}), that is every access to the members are to be enclosed
in synchronized(this) blocks and no potentially blocking operation is to be
invoked within such block.


\subsection{Extra (C)} % Nico
\subsubsection{Fair alley synchronization}

As previously mentioned, our implementations of the alley (based on semaphores
and monitors) are not fair. One solution to implement a fair alley is to ensure
that the children enter the alley in the order of their arrival. It is also
important to still make use of the fact that several children can be in the
alley if they go in the same direction.\\
        ~\\
This can be implemented using a queue. every child coming in enters the queue
and waits for his turn. When a child is at the front of the queue he can go as
soon as no one going in his opposite direction is in the alley.
Programmatically, we implemented the fair alley using a queue of pair
direction/semaphore. The semaphore is here to block the car (and only this
one), while the direction is there for the alley to decide whether the
child needs to wait for someone to come out of the queue or may go in (while
his friends of the same direction are still in the alley).

The source code for the fair alley is available in appenix \ref{fairAlley}
~\\
~\\
This implemetation is not perfect as cars are registered in the queue only when
they reach specific cells of the playfield. The problem is that cars naturally
queue in order to not bump into one another (see section \ref{section1}). As a
result when a car arrives near an entrance of the alley and another is already
waiting there, the former waits behind him without registering to the
alley's entrance queue, until the later enters the alley, and only then the
former will move to the entrace and register to the queue. The resulting
behaviour is still fair, yet it looks less natural as if cars would register to
the queue when they are blocked by a neighboring car.

