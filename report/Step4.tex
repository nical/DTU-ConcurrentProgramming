
\section{Step 4}

\subsection{Monitors} % Nico

The use of Java monitors brings some syntactic sugar to the implementation of
the mutual exclusion when accessing members of the alley/barrier/etc. The
algorithms can be rewritten by replacing:
\begin{verbatim}
[...]
atomicAccess.P();
[...] // access to private members
atomicAccess.V();
[...]
\end{verbatim}
with:
\begin{verbatim}
[...]
synchronized(this) {
[...]  // access to private members
}
[...]
\end{verbatim}

As shown in the monitor based implementations of the alleys and the barrier.
Monitors are convenient to ensure mutual exclusion but do not help with blocking
threads intentionally as the main semaphores of the barrier and the alley do.
Semaphores are better suited for this task, and are kept in the implementation
of the monitor based alley and barrier.

Using monitors follow the same basic rules that with mutual exclusion semaphores
(\texttt{atomicAccess}), that is every access to the members are to be enclosed
in synchronized(this) blocks and no potentially blocking operation is to be
invoked within such block.


\subsection{Extra (C)} % Nico
\subsubsection{Fair alley synchronization}

