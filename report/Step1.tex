


\section{Step 1}

\subsection{Cars bumping into each other} %Jakub

\subsection{The alley} %Nico

The alley is a narrow section of the play field where the children can come two
different directions but there is only room for one direction at the same time.

This situation reflects the problem of a critical section that can be accessed
by several actors of the same type concurrently but only support one type of
actor at any given time.

\subsubsection{Several implementation}

As the subject require several versions of the alley, the implementation was
split in the following way:
\begin{itemize}
    \item The \texttt{AlleyBehaviour} is an interface that corresponds to the
    functions that the alley must fulfill (enter and leave).
    \item Each implementation of the alley (\texttt{SemaphoreAlley},
            \texttt{MonitorAlley} and \texttt{FairAlley}) implement the
            \texttt{AlleyBehaviour} interface.
    \item The \texttt{Alley} class is a simple class that forwards its calls to
    an \texttt{AlleyBehaviour} (member that is given to it in the constructor),
    and provide a method for the cars to know if they are entering or leaving
    the Alley.
\end{itemize}
    
The implementation of the barrier follows the same principle.

\subsubsection{Principle of the algorithm}

In order to implement the alley, a distinction is made between the cars coming
from the top and the cars coming from the bottom when entering the alley (as
well as he car that does not enter the alley at all). This distinction is stored
in the car instances as an integer number computed at the construction of the
car object. The invariant rule of the alley is that only cars of a type T can be
in the alley if the current type of the alley if T. The type is therefore stored
as an integer member of the alley and can be one of the following values: A, B
or FREE (free meaning nobody is in the alley and the next incoming car sets the
current type of the alley). The alley also needs to keep track of the number of
cars it contains by incrementing and decrementing a counter when a car enters
and leaves. The previously mentioned members must not be accessed concurrently,
therefore the alley uses a semaphore (named \texttt{atomicAccess} in the code)
that is used to ensure that only one thread at a time accesses the members. At
last, an extra semaphore is used to block the incoming car threads when they 
have to wait before entering the alley.

To make the algorithm and help making it thread-safe, some simple rules were
observed:
\begin{itemize}
    \item access to the member variables must be protected by a \textit{P}
    operation of the \texttt{atomicAccess} semaphore before the access and a
    V operation of the same semaphore after the call.
    \item The P operation of the main alley semaphore - or any other
    potentially blocking operation - must not be called within a P-V pair 
    of the \texttt{atomicAccess} semaphore.
\end{itemize}

These simple rules are not sufficient to strictly ensure correctness but they
were helpful enough to be mentioned here.

The resulting algorithm is simple: 
\begin{itemize}
    \item Incoming cars first test whether their direction is different from the
    current direction of the alley. If so, they must P the main semaphore which
    leads to two possibilities:
    \begin{itemize}
        \item The alley is empty which means that the semaphore is available and
        the thread is not blocked.
        \item The alley already contains one or several cars and as the current
        direction of the alley is different from the one of the car, it means
        that the alley is occupied by the other direction, and the semaphore is
        already taken, so the thread is blocked.
    \end{itemize}
    After this point, the car is in the alley, the current direction of the
    alley is set to this car's direction and the counter is incremented.
    \item Leaving cars decrement the counter and check whether the later
    reaches zero, in which case the current direction of the alley is set to
    zero and the main semaphore is freed by a V operation
\end{itemize}

\subsubsection{Starvation issues}

The algorithm in this form makes the children wait until there is no car coming
from the other direction in the alley, and lets the ones going in the same
direction enter without waiting. This situation is unfair and can lead to
starvation because it can happen that cars have the enough time to make a full
turn while other cars going in the same direction are still occupying the alley,
and enter again, and so on, while other unfortunate children may be waiting in
the other side. The more active cars in the play field, the most likely the
problem is to happen, and the alley speed reduction makes it, of course, even
worse since it makes car spend even more time in the alley, providing the cars
that are out with more time to circle around and be back in.

